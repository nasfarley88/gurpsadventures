\documentclass[a4paper,twocolumn]{memoir}

\usepackage{xparse}
\usepackage{gurps}

% Fonts
\usepackage{libertine}
\renewcommand*\familydefault{\sfdefault}

\NewDocumentEnvironment{vignette}{}{\noindent\itshape}{\noindent}

\begin{document}

\chapter{Rooms}
\label{cha:rooms}

\subsection{Candy cane golem}
\label{sec:candy-cane-golem}

\begin{vignette}
  The gooey mess heaved as it laid on the floor. As the adventurers approached,
  the mass \emph{stood up}! Towering above the adventurers, 12 feet of candy
  cane formed a huge golem, and roared like the fire of a thousand kilns. 
\end{vignette}

This room consists of an actual kiln and a candy cane golem (below). The candy
cane golem is really a fire elemental trapped inside candy cane and it is
\emph{angry} about it. The heat of the fire elemental is the only way the candy
cane stays mobile. If for any reason the fire elemental loses it's fire/heat,
the golem becomes immobile.


\begin{character}
  \ST{21}
  \DX{12}
  \IQ{8}
  \HT{12}
  \Will{10}
  \SM{2}
  \skill{Brawling}[IQ/Easy]{16}
\end{character}

\subsection{Blizzard room}
\label{sec:blizzard-room}


\subsection{Krampus}
\label{sec:krampus}

\IfFileExists{krampus.tex}{% The work in here is for personal use only. It is from http://www.ravensnpennies.com/2013/12/triple-threat-krampus.html
\begin{character}{ST=35, DX=18, IQ=16, HT=16, SM=2}
% ST: 35                    HP: 35                  Speed: 10.00
% DX: 18                   Will: 20                Move: 16
% IQ: 16                    Per: 20                 Weight: 900 lbs.
% HT: 16                   FP: 21                   SM: +1
% EN: 20                   EP: 20
% Dodge: 14             Parry: 14              DR: 10
\advantage{Appearance (Monstrous)}
\advantage{Bully (6)}
\advantage{Clinging (Move 16)}
\advantage{Code of Honor (Father Christmas’s Companion’s)}
\advantage{Combat Reflexes}
\advantage{Compulsive Castigation (6)}
\advantage{Constriction Attack}
\advantage{Dark Vision (Color Vision; Hypersensory)}
\advantage{Dayblindness*}
\advantage{Detect\footnote{(Naughtiness; Analyzing; Cosmic, Ignores Countermeasures;
  Cosmic, No die roll required; Cosmic, Privileged Ability; Increased Range,
  Line-of-Sight; Nuisance Effect, Only the past calendar year; Precise;
  Reflexive; Reliable 10; Time-Spanning, Past \& Future; World-Spanning, All)}}
\advantage{Divine Curse (Cannot Harm Innocents)}
\advantage{Divine Curse (Must Return To The North Pole When Christmas Is Over)}
\advantage{Doesn’t Breathe}
\advantage{Doesn’t Sleep}
\advantage{Double-Jointed}
\advantage{Extra Arm 1 (Extra-Flexible}
\advantage{Long, +2 SM)}
\advantage{Extra Attack 2 (Multi-Strike)}
\advantage{Extreme Fanaticism (Father Christmas)}
\advantage{Filter Lungs}
\advantage{Frightens Animals}
\advantage{Higher Purpose 3 (Punish the naughty)}
\advantage{Immunity to Metabolic Hazards}
\advantage{Immunity to Mind Control}
\advantage{Indomitable (Cosmic)}
\advantage{Injury Tolerance (Humongous; Damage Reduction 2; Unbreakable Bones)}
\advantage{Intolerance (`Naughty' children and people)}
\advantage{Jumper\footnote{(Spirit; Extra Carrying Capacity, Extra-Heavy; No
  Concentration; No Strain; Reduced Essence, 5 EP; Reliable 10; Special Portal,
  chimneys, stoves, fireplaces, etc.; Tracking; Tunnel, Forms Before
  Teleportation; Warp Jump)}}
\advantage{Long Arms (SM +1)}
\advantage{Magic Resistance 10 (Improved; Switchable)}
\advantage{Nocturnal}
\advantage{Odious Personal Habit (Eats sapients)}
\advantage{Overconfidence (12)}
\advantage{Perfect Balance}
\advantage{Peripheral Vision}
\advantage{Regeneration (Extreme; 30 HP/second)}
\advantage{Restricted Diet (Human Flesh; Substitution, Animal Flesh)}
\advantage{Sadism (6)}
\advantage{Shadow Form\footnote{(3-D Movement; Affect Insubstantial; Can Carry Objects,
  Extra Heavy; Insubstantial; Maximum Duration, 1d seconds; Partial Change;
  Takes Recharge, 1d+1 seconds)}}
\advantage{Social Stigma (Monster)}
\advantage{Super Climbing 8}
\advantage{Super Jump 3}
\advantage{Supernatural Durability (Hawthorn wood varnished with mistletoe berries)}
\advantage{Supernatural Feature (Omens)}
\advantage{Terror 5† (Area Effect, 40 yards; Presence)}
\advantage{Trademark (Leaves lumps of coal in the homes of those he’s kidnapped)}
\advantage{Unfazeable (Cosmic)}
\advantage{Unkillable 3 (Trigger, The Calendar Month of December)}
\advantage{Vulnerability (Blinding Attacks x2)}
\advantage{Warp\footnote{(Blind; Extra Carrying Capacity, Extra-Heavy; No
    Strain; Reduced Essence, 5 EP; Reliable 10; Special Portal, chimneys, stoves,
    fireplaces, etc.; Tracking; Tunnel, Forms Before Teleportation; Warp Jump)}}
\advantage{Weapon Master (Birch Branch)}
\skill{Acrobatics}{20}
\skill{Brawling}{20}
\skill{Climbing}{25}
\skill{Escape}{25}
\skill{Hidden Lore (Faerie)}{20}
\skill{Innate Attack (Projectile)}{20}
\skill{Intimidate}{25}
\skill{Invisibility Art}{25}    % ‡
\skill{Jumping}{20}
\skill{Observation}{20}
\skill{Stealth}{20}
\skill{Survival (Arctic)}{20}
\skill{Swimming}{20}
\skill{Throwing}{20}
\skill{Whip}{20}
\skill{Wrestling}{20}

\end{character}

\paragraph{Basket Binding (25)} Krampus can place those it has already grappled or pinned in the basket he wears on his back. This requires a Quick Contest his ST 35 against the best of their DX, ST, Escape, or grappling skill. Failure means they are inside the basket and must make a roll against the lower of their EN, HT, or Will vs. this ability’s skill or be paralyzed for hours equal to Krampus’ margin of success..

\paragraph{Birch Switch (18)} \dice{6}{+11}(0.5) crushing linked Afflicition-5(2). The affliction causes Terrible Pain giving those who fail a -6 DX, IQ, skill, and self-control rolls. Reach C-3. Cannot Parry.  Made as a Deceptive Attack (-2 to defend against).

\paragraph{Bite (18)} \dice{4}{+3} cutting; Reach C-1. Made as a Deceptive Attack (-2 to defend against).

\paragraph{Bite, Running (15)} \dice{4}{+3} cutting; Reach C-1. Made as a Move and Attack; ignore the skill cap of 9.

\paragraph{Chain Rattle (20)} No damage. Acc 3. Targets afflicted have a -10 to all Hearing rolls, he is effectively deaf! This affects an area of up to 8-yards originating from the point of impact. This lasts for 10 seconds. Range −/100. Krampus is not affected by this attack.

\paragraph{Claw (16)} \dice{4}{+3} impaling or cutting; Reach C-2. Made as a Deceptive Attack (-3 to defend against).

\paragraph{Claw, Running (15)} \dice{4}{+3} impaling or cutting; Reach C-2. Made as a Move and Attack; ignore the skill cap of 9.

\paragraph{Coal Bomb (16)} \dice{3}{} [1d(0.2) burning] crushing explosion incendiary. Acc 3. Range 50/100. RoF 1,
Shots 1 (T), Recoil 1, Bulk -2.

\paragraph{Double-Claw (16)} \dice{4}{+3} impaling or cutting; Reach C-2. Made as a Deceptive Attack (-1 to defend against). Targets two adjacent foes simultaneously as a single attack. Alternatively, Krampus may grab and then grapple a target. (May only make one double-claw per turn.)

\paragraph{Improvised Weapon (13)} Based on Damage \dice{4}{-1}/\dice{6}{+1}.

\paragraph{Ink Squirt (20)} No damage. Acc 3. Targets afflicted have a -10 to all Vision rolls, he is effectively blind! This affects an area of up to 8-yards originating from the point of impact. This lasts for 10 seconds. Range −/100. Krampus is not affected by this attack.

\paragraph{Tongue Lashing (16)} \dice{4}{+7} cutting or crushing. Reach C-3. Treat as a weapon (Striker), not as a body part. Alternatively, Krampus may grapple a target with it. Made as a Deceptive Attack (-2 to defend against).

\paragraph{Torso Grapple (20)} No damage, but on further turns can squeeze (Choke or Strangle, p. B370) as ST 35. Treat this as a two-handed grapple. Krampus can use his tongue to help, giving him a +2 bonus to rolls.
%%% Local Variables:
%%% mode: latex
%%% TeX-master: "main"
%%% End:
}{}


\end{document}

%%% Local Variables:
%%% mode: latex
%%% TeX-master: t
%%% End:
