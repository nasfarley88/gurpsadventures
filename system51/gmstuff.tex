\chapter{Adventure prep}
\label{cha:adventure-prep}

\section{General ideas}
\label{sec:general-ideas}

\section{Nuclear crater}
\label{sec:nuclear-crater}

Due to an explosion of the nuclear variety, the adventurers are sent into the
large crater\ldots{}.

\subsection{Hooks}
\label{sec:hooks-1}

\subsubsection{\emph{Aliens!}}
\label{sec:emphaliens}

The crater was caused by alien contact and there is a surviving alien at the
centre that they must retrieve alive (or for certain other contractors who will
contact the characters individually, dead for a higher fee).

\subsubsection{Evidence collection}
\label{sec:evidence-collection}

The company responsible for the explosion doesn't want it's involvement getting
out. The team need to get in and out taking or destroying any evidence with
them.

Of course, they are not the only ones after the loot. Battle may ensue, made
even worse by radioactive dust kicked up by every explosion.

\subsubsection{\thecompany requires\ldots{}}
\label{sec:thec-requ}

\thecompany needs the technology used to create that blast, so it's sending in
it's best team! Or at least it's most expendable\ldots{}

(This is very similar to \ref{sec:evidence-collection} but the end goal is
different.)

\section{Session 1}

\subsection{Hooks}
\label{sec:hooks}

\subsubsection{The angry boss}
\label{sec:angry-boss}

\paragraph{Basic idea}

Players are told beforehand that something is going to happen and they are going
to have to improv. Whatever they say will be accepted as true (or a lie if they
want). The characters then enter a room where the boss shouts ``What the hell
happened? You killed the wrong guy!'' \emph{Cue improv.}

\paragraph{Pros}

\begin{itemize}
\item Don't need to know much about the characters
\item Shows people's colours quickly
\end{itemize}

\paragraph{Cons}

\begin{itemize}
\item Doesn't lead anywhere by default
\end{itemize}


\subsubsection{The job}
\label{sec:job}

\paragraph{Basic idea}

The team get given a job, Charlie's Angels style: rob a mafia bank. Payout is
given \emph{on delivery} but only a certain amount must be stolen (client's
wishes). If more is stolen the team will be in trouble.

\paragraph{Pros}

\begin{itemize}
\item \player{Thief} will totally steal more
\item Can split the party to go information searching
  \begin{itemize}
  \item This even means that David and I can find different information and
    combine it (making more improv.\ excellency)
  \end{itemize}
\end{itemize}

\paragraph{Cons}

Will have to prep a building for everyone to get involved in. Basically design
an entire challenge.

\subsubsection{Kidnapped daughter of a CEO}
\label{sec:kidn-daught-ceo}

As Google Doc.


\chapter{Reusable NPCs}
\label{cha:reusable-npcs}

\section{Armed mook}
\label{sec:armed-mook}

Simple enough. Use this template for any NPC who is prepared for a fight but
the GM hasn't prepared them for it. They will die quickly and will probably
immediately fail any HT roll for falling unconcious/dying.

They have the following weapon:
\begin{innateattack}
  \item Needs \dice{3} of damage so make it \InnateAttackLevel{3}
  \item Attack type pi means \InnateAttackPointsPerLevel{5}
  \item No point in extending the range (can assume infinite, won't be shooting
    far anyway!)
  \item That makes this weapon \InnateAttackTotalPoints.
\end{innateattack}

\begin{character}
  \skill{Brawling}[DX/Easy]{10}
  \DR{5}
  \advantage{Innate Attack (pistol, \dice{3}, pi)}[15]
  \skill{`Innate' Attack (pistol)}{10}
\end{character}

\chapter{Weapons}
\label{cha:weapons}

\section{Cordite pistol}
\label{sec:cordite-pistol}

Assumes a cordite pistol with \dice{3} damage

\begin{innateattack}
  \item Cordite pistols are pi damage so \InnateAttackPointsPerLevel{5} with
    \dice{3} damage for \InnateAttackLevel{3}.
  \item Clip of 10 means \InnateAttackPercentModifier{-10} with Limited Use
  \item Recoil is pretty bad for the base (Rcl 3) so
    \InnateAttackPercentModifier{-20}
  \item Accuracy needs taking down a little to 2 (so that rifles are more
    accurate than pistols) so \InnateAttackPercentModifier{-5}
  \item It's \emph{loud} which means a single level of Nuisance Effect at
    \InnateAttackPercentModifier{-5}
  \item This means the cordite pistol is worth \InnateAttackTotalPoints{}
\end{innateattack}

\section{Cordite rifle}
\label{sec:cordite-rifle}

Assumes a cordite rifle with \dice{6} damage

\begin{innateattack}
  \item Cordite rifles are pi damage so \InnateAttackPointsPerLevel{5} with
    \dice{6} damage for \InnateAttackLevel{6}.
  \item Clip of 15 means Limited Use but no modifiers
  \item Recoil is OK for the base (Rcl 2) so \InnateAttackPercentModifier{-10}
  \item Accuracy at 3 is fine
  \item It's \emph{loud} which means a single level of Nuisance Effect at
    \InnateAttackPercentModifier{-5}
  \item This means the cordite rifle is worth \InnateAttackTotalPoints{}
\end{innateattack}


\section{Laser pistol}
\label{sec:laser-pistol}

Assumes a laser pistol with same point spend as cordite pistol but more damage

\begin{innateattack}
  \item Laser pistols are pi- damage so \InnateAttackPointsPerLevel{3} with
    \dice{3} damage for \InnateAttackLevel{3}.
  \item Clip is effectively unlimited
  \item Recoil is 1
  \item Accuracy needs taking down a little to 2 (so that rifles are more
    accurate than pistols) so \InnateAttackPercentModifier{-5}
  \item It's silent
  \item This means the laser pistol is worth \InnateAttackTotalPoints{}
\end{innateattack}


\section{Laser rifle}
\label{sec:laser-rifle}

Assumes a laser rifle matches points spend with cordite rifle, but has more damage

\begin{innateattack}
  \item Laser rifles are pi- damage so \InnateAttackPointsPerLevel{3} with
    \dice{9} damage for \InnateAttackLevel{9}.
  \item Clip is effectively unlimited (>200 shots)
  \item Recoil is 1 so no modification
  \item Accuracy at 3 is fine
  \item It's silent
  \item This means the laser rifle is worth \InnateAttackTotalPoints{}
\end{innateattack}

%%% Local Variables:
%%% mode: latex
%%% TeX-master: "main"
%%% End:
